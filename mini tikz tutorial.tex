%basic formatting: paper size, margin size, font size, language, etc.
\documentclass[11pt]{article}
\usepackage[english]{babel}
\usepackage[letterpaper,margin=1in]{geometry} 

\usepackage{multicol} %for multiple columns, so that I can display code and output side-by-side

%package to draw trees, with accompanying "decorations"
\usepackage{tikz} 
\usetikzlibrary{arrows,shapes,trees,positioning,arrows.meta}

%font packages. The Charis SIL font allows you to input symbols like IPA symbols directly (as opposed to using a package like textipa).
%Very useful IPA keyboard: https://westonruter.github.io/ipa-chart/keyboard/
%IPA and other language-specific symbols: https://ipa.typeit.org/

%TO USE THIS FONT, YOU MUST COMPILE YOUR FILE WITH XELATEX. On Overleaf, go to Menu on the upper left corner and choose this option on Compiler.

\usepackage{fontspec} 
\usepackage{xltxtra} 
\setmainfont{Charis SIL} 
\usepackage{xunicode} 

\usepackage{pifont} %for symbols

%package for underlining, crossing out, etc.
\usepackage[normalem]{ulem} 

%clickable cross-references (for numbered examples and references) 
\usepackage[hidelinks]{hyperref}

%numbered examples
\usepackage{gb4e}

%other packages for numbered examples: linguex, expex

\begin{document}

\title{\textbf{Mini \texttt{tikz} tutorial}}
\author{Suzana Fong\\
\href{mailto:sznfong@alum.mit.edu}{sznfong@alum.mit.edu}}
\date{\today}
\maketitle

In this short tutorial, I list some basic trees and some basic decorations like different types of arrows. I learned a lot from this phenomenal tutorial by Dr. James Crippen: \url{https://lingbuzz.net/lingbuzz/003379}. A lot of the code in this tutorial comes from this manual. Things I learned overt the years from \url{https://tex.stackexchange.com/} are impossible to list and credit.

I claim no credit nor novelty in this tutorial. It just conveniently lists codes that are frequently used in syntax documents.

\begin{enumerate}
    \item A basic head-initial tree
    
    \begin{multicols}{2}
    
    \begin{verbatim}
\begin{tikzpicture}
    [parent anchor=south,
    align=center,
    level distance=2.25em,
    anchor=north,
    sibling distance=6em,
    child anchor=north]
    \node {A}
    child {node {B}}
    child {node {C}
        child {node {D}}
        child {node {E}
            child {node {F}}
            child {node {G}}}}
    ;
\end{tikzpicture}
        \end{verbatim}
        
        \columnbreak
        
        \begin{tikzpicture}
        [parent anchor=south,
    	align=center,
    	level distance=2.25em,
    	anchor=north,
    	sibling distance=6em,
    	child anchor=north]
    	\node {A}
    		child {node {B}}
    		child {node {C}
    		    child {node {D}}
    		    child {node {E}
    		        child {node {F}}
    		        child {node {G}}}}
    		
    			;
    	\end{tikzpicture} 
    \end{multicols}
    
    Adjust the vertical distance between nodes with \texttt{level distance=2.25em} and the horizontal distance with \texttt{sibling distance=2.25em}.
    
    \pagebreak
    
    \item A basic head-final tree
    
    \begin{multicols}{2}
    
    \begin{verbatim}
\begin{tikzpicture}
    [parent anchor=south,
    align=center,
    level distance=2.25em,
    anchor=north,
    sibling distance=6em,
    child anchor=north]
    \node {A}
    child {node {C}
        child {node {E}
            child {node {G}}
            child {node {F}}}
        child {node {D}}}
    child {node {B}}
    ;
\end{tikzpicture}
        \end{verbatim}
        
        \columnbreak
        
        \begin{tikzpicture}
        [parent anchor=south,
    	align=center,
    	level distance=2.25em,
    	anchor=north,
    	sibling distance=5em,
    	child anchor=north]
    	\node {A}
    	child {node {C}
    	    child {node {E}
    	        child {node {G}}
    	        child {node {F}}}
    	    child {node {D}}}
    	child {node {B}}
    	;
    	\end{tikzpicture} 
    \end{multicols}
    
    \item A tree without labels in branching nodes:
    
    \begin{multicols}{2}
    
    \begin{verbatim}
\begin{tikzpicture}
    [parent anchor=center,
    align=center,
    level distance=2em,
    anchor=north,
    sibling distance=5em,
    child anchor=north]
    \node {}
    child {node {B}}
    child {
        child {node {D}}
        child {
            child {node {F}}
            child {node {G}}}}
    ;
\end{tikzpicture}
        \end{verbatim}
        
        \columnbreak
        
        \begin{tikzpicture}
        [parent anchor=center,
    	align=center,
    	level distance=2em,
    	anchor=north,
    	sibling distance=6em,
    	child anchor=north]
    	\node {}
    	child {node {B}}
    	child {
    	    child {node {D}}
    	    child {
    	        child {node {F}}
    		    child {node {G}}}}
    			;
    	\end{tikzpicture} 
    \end{multicols}
    
    Key specification: \texttt{parent anchor=center}. Then delete \texttt{node \{\}} in a branching node.
    
    \pagebreak
    
    \item A curved arrow connecting nodes G and B
    
    \begin{multicols}{2}
    
    \begin{verbatim}
\begin{tikzpicture}
    [parent anchor=south,
    align=center,
    level distance=2.25em,
    anchor=north,
    sibling distance=6em,
    child anchor=north]
    \node {A}
    	child {node (b) {B}}
    	child {node {C}
    	    child {node {D}}
    	    child {node {E}
    	        child {node {F}}
    	        child {node (g) {G}}}}
    	;
    	\draw[<-] (b) .. controls +(south:7em) and +(south:5em) .. (g);
\end{tikzpicture}
        \end{verbatim}
        
        \columnbreak
        
        \begin{tikzpicture}
        [parent anchor=south,
    	align=center,
    	level distance=2.25em,
    	anchor=north,
    	sibling distance=6em,
    	child anchor=north]
    	\node {A}
    		child {node (b) {B}}
    		child {node {C}
    		    child {node {D}}
    		    child {node {E}
    		        child {node {F}}
    		        child {node (g) {G}}}}
    	;
    	\draw[<-] (b) .. controls +(south:7em) and +(south:5em) .. (g);
    	\end{tikzpicture} 
    \end{multicols}
    
    Add labels to the nodes you want to connect with an arrow (e.g. \texttt{(b)} and \texttt{(g)}).
    
    \item Another code for a curved arrow
    
    \begin{multicols}{2}
    
    \begin{verbatim}
\begin{tikzpicture}
    [parent anchor=south,
    align=center,
    level distance=2.25em,
    anchor=north,
    sibling distance=6em,
    child anchor=north]
    \node {A}
    	child {node (b) {B}}
    	child {node {C}
    	    child {node {D}}
    	    child {node {E}
    	        child {node {F}}
    	        child {node (g) {G}}}}
    	;
    	\draw[<-] (b) to [bend right=75] (g.south);
\end{tikzpicture}
        \end{verbatim}
        
        \columnbreak
        
        \begin{tikzpicture}
        [parent anchor=south,
    	align=center,
    	level distance=2.25em,
    	anchor=north,
    	sibling distance=6em,
    	child anchor=north]
    	\node {A}
    		child {node (b) {B}}
    		child {node {C}
    		    child {node {D}}
    		    child {node {E}
    		        child {node {F}}
    		        child {node (g) {G}}}}
    	;
    	\draw[<-] (b) to [bend right=75] (g.south);
    	\end{tikzpicture} 
    \end{multicols}
    
    \pagebreak
    
    \item An angled arrow connecting B and G
    
    \begin{multicols}{2}
    
    \begin{verbatim}
\begin{tikzpicture}
    [parent anchor=south,
    align=center,
    level distance=2.25em,
    anchor=north,
    sibling distance=6em,
    child anchor=north]
    \node {A}
    	child {node (b) {B}}
    	child {node {C}
    	    child {node {D}}
    	    child {node {E}
    	        child {node {F}}
    	        child {node (g) {G}}}}
    	;
    	\draw[<-,rounded corners=.25em] (b.south)--+(0,-85pt)-|(g.south);
\end{tikzpicture}
        \end{verbatim}
        
        \columnbreak
        
        \begin{tikzpicture}
        [parent anchor=south,
    	align=center,
    	level distance=2.25em,
    	anchor=north,
    	sibling distance=6em,
    	child anchor=north]
    	\node {A}
    		child {node (b) {B}}
    		child {node {C}
    		    child {node {D}}
    		    child {node {E}
    		        child {node {F}}
    		        child {node (g) {G}}}}
    	;
    	\draw[<-] (b.south)--+(0,-85pt)-|(g.south);
    	\end{tikzpicture} 
    \end{multicols}
    
    \item Other arrow tips and useful commands to be used in \texttt{draw[\dots]}:
    
    \begin{enumerate}
        \item \texttt{*-*} (filed circles)
        \item \texttt{o-o} (circle lines)
        \item \texttt{latex-} (different arrow style)
        \item \texttt{rounded corners=.25em} (rounded corners for angled arrow)
        \item \texttt{dashed} (dashed line instead of solid line)
    \end{enumerate}
    
    \pagebreak
    
    \item Annotation \textit{hello} added to curved arrow
    
    \begin{multicols}{2}
    
    \begin{verbatim}
\begin{tikzpicture}
    [parent anchor=south,
    align=center,
    level distance=2.25em,
    anchor=north,
    sibling distance=6em,
    child anchor=north]
    \node {A}
    	child {node (b) {B}}
    	child {node {C}
    	    child {node {D}}
    	    child {node {E}
    	        child {node {F}}
    	        child {node (g) {G}}}}
    	;
    	\draw[<-] (b) .. controls +(south:7em) and +(south:5em) .. (g) 
    	node [anchor=center,pos=0.5,fill=white] {hello};
\end{tikzpicture}
        \end{verbatim}
        
        \columnbreak
        
        \begin{tikzpicture}
        [parent anchor=south,
    	align=center,
    	level distance=2.25em,
    	anchor=north,
    	sibling distance=6em,
    	child anchor=north]
    	\node {A}
    		child {node (b) {B}}
    		child {node {C}
    		    child {node {D}}
    		    child {node {E}
    		        child {node {F}}
    		        child {node (g) {G}}}}
    	;
    	\draw[<-] (b) .. controls +(south:7em) and +(south:5em) .. (g)
    	node [anchor=center,pos=0.5,fill=white] {hello};
    	\end{tikzpicture} 
    \end{multicols}
    
    Instead of \textit{hello}, you can use a symbol like \ding{55}. You will need this package in the preamble: 
    
    \begin{verbatim}
\usepackage{pifont}
    \end{verbatim}
    
    \item Annotation \textit{hello} added to angled arrow
    
    \begin{multicols}{2}
    
    \begin{verbatim}
\begin{tikzpicture}
    [parent anchor=south,
    align=center,
    level distance=2.25em,
    anchor=north,
    sibling distance=6em,
    child anchor=north]
    \node {A}
    	child {node (b) {B}}
    	child {node {C}
    	    child {node {D}}
    	    child {node {E}
    	        child {node {F}}
    	        child {node (g) {G}}}}
    	;
    	\draw[<-,rounded corners=.25em] (b.south)--+(0,-85pt)-|(g.south)
    	node [anchor=center,pos=0.25,fill=white] {hello};
\end{tikzpicture}
        \end{verbatim}
        
        \columnbreak
        
        \begin{tikzpicture}
        [parent anchor=south,
    	align=center,
    	level distance=2.25em,
    	anchor=north,
    	sibling distance=6em,
    	child anchor=north]
    	\node {A}
    		child {node (b) {B}}
    		child {node {C}
    		    child {node {D}}
    		    child {node {E}
    		        child {node {F}}
    		        child {node (g) {G}}}}
    	;
    	\draw[<-] (b.south)--+(0,-85pt)-|(g.south)
    	node [anchor=center,pos=0.25,fill=white] {hello};
    	\end{tikzpicture} 
    \end{multicols}
    
    \item Annotation \textit{hello} added beside curved arrow
    
    \begin{multicols}{2}
    
    \begin{verbatim}
\begin{tikzpicture}
    [parent anchor=south,
    align=center,
    level distance=2.25em,
    anchor=north,
    sibling distance=6em,
    child anchor=north]
    \node {A}
    	child {node (b) {B}}
    	child {node {C}
    	    child {node {D}}
    	    child {node {E}
    	        child {node {F}}
    	        child {node (g) {G}}}}
    	;
    	\draw[<-] (b) .. controls +(south:7em) and +(south:5em) .. (g) 
    	node [anchor=center,pos=0.5,fill=white] {hello};
\end{tikzpicture}
        \end{verbatim}
        
        \columnbreak
        
        \begin{tikzpicture}
        [parent anchor=south,
    	align=center,
    	level distance=2.25em,
    	anchor=north,
    	sibling distance=6em,
    	child anchor=north]
    	\node {A}
    		child {node (b) {B}}
    		child {node {C}
    		    child {node {D}}
    		    child {node {E}
    		        child {node {F}}
    		        child {node (g) {G}}}}
    	;
    	\draw[<-] (b) .. controls +(south:7em) and +(south:5em) .. (g)
    	node[sloped,midway] {hello};
    	\end{tikzpicture} 
    \end{multicols}
    
    Instead of \textit{hello}, you can use a symbol like \ding{55}. For this particular symbol, you will need this package in the preamble: 
    
    \begin{verbatim}
\usepackage{pifont}
    \end{verbatim}
    
    \pagebreak
    
    \item Annotation \textit{hello} added below angled arrow
    
    \begin{multicols}{2}
    
    \begin{verbatim}
\begin{tikzpicture}
    [parent anchor=south,
    align=center,
    level distance=2.25em,
    anchor=north,
    sibling distance=6em,
    child anchor=north]
    \node {A}
    	child {node (b) {B}}
    	child {node {C}
    	    child {node {D}}
    	    child {node {E}
    	        child {node {F}}
    	        child {node (g) {G}}}}
    	;
    	\draw[<-,rounded corners=.25em] (b.south)--+(0,-85pt)-|(g.south)
    	node [anchor=north,pos=0.25,yshift=-.25em] {hello};
\end{tikzpicture}
        \end{verbatim}
        
        \columnbreak
        
        \begin{tikzpicture}
        [parent anchor=south,
    	align=center,
    	level distance=2.25em,
    	anchor=north,
    	sibling distance=6em,
    	child anchor=north]
    	\node {A}
    		child {node (b) {B}}
    		child {node {C}
    		    child {node {D}}
    		    child {node {E}
    		        child {node {F}}
    		        child {node (g) {G}}}}
    	;
    	\draw[<-] (b.south)--+(0,-85pt)-|(g.south)
    	node [anchor=north,pos=0.25,yshift=-.25em] {hello};
    	\end{tikzpicture} 
    \end{multicols}
    
    \item Linear representation with arrows
    
    \begin{verbatim}
\tikzstyle{every picture}+=[remember picture, inner sep=0pt, baseline, anchor=base]%
	
	{}[stuff \tikz\node(b){B}; [more stuff [even more stuff \tikz\node(g){G};]]]
	
	\begin{tikzpicture}[overlay]
	    \draw[<-,rounded corners=.25em]([yshift=-.5em]b.south)--+(0,-1.5em)-|
	    ([yshift=-.5em]g.south);
    \end{tikzpicture}        
    \end{verbatim}
    
    \tikzstyle{every picture}+=[remember picture, inner sep=0pt, baseline, anchor=base]%
	
	{}[stuff \tikz\node(b){B}; [more stuff [even more stuff \tikz\node(g){G};]]]
	
	\begin{tikzpicture}[overlay]
	    \draw[<-]([yshift=-.5em]b.south)--+(0,-1.5em)-|
	    ([yshift=-.5em]g.south);
    \end{tikzpicture}
    
    \begin{itemize}
        \item The element E in the linear representation you want to connect with an arrow must be placed in this code below. Note the label \texttt{(e)} and the semicolon \texttt{;}.
        
        \begin{verbatim}
\tikz\node(e){E};
        \end{verbatim}
       
       \item The code below places the arrow lower down the elements it connects.
       
       \begin{verbatim}
[yshift=-.5em]
       \end{verbatim}
        
    \end{itemize}
    
\end{enumerate}

\section*{Other useful resources}

\begin{enumerate}
    \item Lining up top of tree with number for numbered exampled: use the following code before the tree:
    
    \begin{verbatim}
\leavevmode\vadjust{\vspace{-\baselineskip}}\newline
    \end{verbatim}
    
    For example, using \texttt{gb4e} to number examples:
    
    \begin{verbatim}
\begin{exe}
    \ex{\leavevmode\vadjust{\vspace{-\baselineskip}}\newline
    \begin{tikzpicture}
    [parent anchor=south,
    align=center,
    level distance=2.25em,
    anchor=north,
    sibling distance=6em,
    child anchor=north]
    \node {A}
    child {node {B}}
    child {node {C}}
    ;
    \end{tikzpicture}}
\end{exe}    
    \end{verbatim}
    
    \item Introducing symbols and special characters: this document uses a \texttt{fontspec} font. It allows one to input symbols like IPA symbols directly, e.g. [ɸβθðʃʒʂʐ].
    
    \begin{itemize}
        \item This is a useful IPA keyboard: \url{https://westonruter.github.io/ipa-chart/keyboard/}.
        \item This website has symbols used in few different writing systems, e.g. Portuguese: \url{https://portuguese.typeit.org/}.
    \end{itemize}
    
    \item Multiple columns with \texttt{minipage}:\footnote{A straightforward package is \texttt{multicol}, but \texttt{minipage} allows for more fine-grained control of the size of the columns.}
    
    \begin{verbatim}
\begin{minipage}[t]{.5\textwidth}
    STUFF IN FIRST COLUMN
\end{minipage}%
\begin{minipage}[t]{.5\textwidth}
    STUFF IN SECOND COLUMN
\end{minipage}%
    \end{verbatim}
    
\end{enumerate}

\end{document}